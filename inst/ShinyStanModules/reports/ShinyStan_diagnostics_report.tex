\documentclass[11pt,]{article}
\usepackage{lmodern}
\usepackage{amssymb,amsmath}
\usepackage{ifxetex,ifluatex}
\usepackage{fixltx2e} % provides \textsubscript
\ifnum 0\ifxetex 1\fi\ifluatex 1\fi=0 % if pdftex
  \usepackage[T1]{fontenc}
  \usepackage[utf8]{inputenc}
\else % if luatex or xelatex
  \ifxetex
    \usepackage{mathspec}
  \else
    \usepackage{fontspec}
  \fi
  \defaultfontfeatures{Ligatures=TeX,Scale=MatchLowercase}
\fi
% use upquote if available, for straight quotes in verbatim environments
\IfFileExists{upquote.sty}{\usepackage{upquote}}{}
% use microtype if available
\IfFileExists{microtype.sty}{%
\usepackage{microtype}
\UseMicrotypeSet[protrusion]{basicmath} % disable protrusion for tt fonts
}{}
\usepackage[margin=1in]{geometry}
\usepackage{hyperref}
\hypersetup{unicode=true,
            pdftitle={ShinyStan Automated Diagnostics Report},
            pdfborder={0 0 0},
            breaklinks=true}
\urlstyle{same}  % don't use monospace font for urls
\usepackage{graphicx,grffile}
\makeatletter
\def\maxwidth{\ifdim\Gin@nat@width>\linewidth\linewidth\else\Gin@nat@width\fi}
\def\maxheight{\ifdim\Gin@nat@height>\textheight\textheight\else\Gin@nat@height\fi}
\makeatother
% Scale images if necessary, so that they will not overflow the page
% margins by default, and it is still possible to overwrite the defaults
% using explicit options in \includegraphics[width, height, ...]{}
\setkeys{Gin}{width=\maxwidth,height=\maxheight,keepaspectratio}
\IfFileExists{parskip.sty}{%
\usepackage{parskip}
}{% else
\setlength{\parindent}{0pt}
\setlength{\parskip}{6pt plus 2pt minus 1pt}
}
\setlength{\emergencystretch}{3em}  % prevent overfull lines
\providecommand{\tightlist}{%
  \setlength{\itemsep}{0pt}\setlength{\parskip}{0pt}}
\setcounter{secnumdepth}{0}
% Redefines (sub)paragraphs to behave more like sections
\ifx\paragraph\undefined\else
\let\oldparagraph\paragraph
\renewcommand{\paragraph}[1]{\oldparagraph{#1}\mbox{}}
\fi
\ifx\subparagraph\undefined\else
\let\oldsubparagraph\subparagraph
\renewcommand{\subparagraph}[1]{\oldsubparagraph{#1}\mbox{}}
\fi

%%% Use protect on footnotes to avoid problems with footnotes in titles
\let\rmarkdownfootnote\footnote%
\def\footnote{\protect\rmarkdownfootnote}

%%% Change title format to be more compact
\usepackage{titling}

% Create subtitle command for use in maketitle
\providecommand{\subtitle}[1]{
  \posttitle{
    \begin{center}\large#1\end{center}
    }
}

\setlength{\droptitle}{-2em}

  \title{ShinyStan Automated Diagnostics Report}
    \pretitle{\vspace{\droptitle}\centering\huge}
  \posttitle{\par}
    \author{}
    \preauthor{}\postauthor{}
      \predate{\centering\large\emph}
  \postdate{\par}
    \date{29 October 2019, 16:10}


\begin{document}
\maketitle

{
\setcounter{tocdepth}{2}
\tableofcontents
}
\hypertarget{warnings}{%
\section{Warnings}\label{warnings}}

\begin{verbatim}
## [1] "65 of 4000 iterations ended with a divergence (1.6%)."
\end{verbatim}

\newpage

\hypertarget{numerical-diagnostics}{%
\section{Numerical diagnostics}\label{numerical-diagnostics}}

\begin{verbatim}
##                 n_eff Rhat   mean se_mean   sd
## log-posterior  261.50 1.02 -19.02    0.30 4.88
## tau            536.96 1.01   7.86    0.25 5.78
## theta[7]       989.39 1.00  11.48    0.22 6.96
## theta[1]      1022.31 1.00  12.46    0.27 8.67
## mu            1029.55 1.00   8.36    0.17 5.43
## theta[5]      1351.81 1.00   4.66    0.17 6.39
## theta[6]      1744.92 1.00   6.10    0.17 7.06
## theta[2]      1840.22 1.00   8.08    0.16 6.70
## theta[3]      2002.39 1.00   6.17    0.19 8.35
## theta[8]      2051.05 1.00   8.92    0.18 8.33
## theta[4]      2051.77 1.00   7.95    0.15 6.77
\end{verbatim}

\newpage

\hypertarget{visual-diagnostics}{%
\section{Visual diagnostics}\label{visual-diagnostics}}

\hypertarget{divergence-information}{%
\subsection{Divergence Information}\label{divergence-information}}

These are plots of the \emph{divergent transition status} (x-axis)
against the \emph{log-posterior} (y-axis top panel) and against the
\emph{acceptance statistic} (y-axis bottom panel) of the sampling
algorithm for all chains. Divergent transitions can indicate problems
for the validity of the results. A good plot would show no divergent
transitions. If the divergent transitions show the same patern as the
non divergent transitions, this could indicate that the divergent
transitions are false positives. A bad plot would shows systematic
differences between the divergent transitions and non-divergent
transitions. For more information see
\url{https://mc-stan.org/misc/warnings.html\#divergent-transitions-after-warmup}.

\includegraphics{ShinyStan_diagnostics_report_files/figure-latex/divergence_transition_plot-1.pdf}

\newpage

\hypertarget{energy}{%
\subsection{Energy}\label{energy}}

These are plots of the overlaid histograms of the marginal energy
distribution (\(\pi_E\)) and the energy transition distribution
(\(\pi_{\Delta E}\)) for all chains. A good plot shows histograms that
look well-matched indicating that the Hamiltonian Monte Carlo should
perform robustly. The closer \(\pi_{\Delta E}\) is to \(\pi_E\) the
faster the random walk explores the energies and the smaller the
autocorrelations will be in the chain. If \(\pi_{\Delta E}\) is narrower
than \(\pi_E\) the random walk is less effective and autocorrelations
will be larger. Additionally the chain may not be able to completely
explore the tails of the target distribution. See Betancourt
\href{https://arxiv.org/abs/1701.02434}{`A conceptual introduction to
Hamiltonian Monte Carlo'} and Betancourt
\href{https://arxiv.org/abs/1604.00695}{`Diagnosing suboptimal cotangent
disintegrations in Hamiltonian Monte Carlo'} for the general theory
behind the energy plots.

\includegraphics{ShinyStan_diagnostics_report_files/figure-latex/energy_plot-1.pdf}

\newpage

\hypertarget{treedepth-information}{%
\subsection{Treedepth Information}\label{treedepth-information}}

These are plots of the \emph{treedepth} (x-axis) against the
\emph{log-posterior} (y-axis top left panel) and against the
\emph{acceptance statistic} (y-axis top right panel) of the sampling
algorithm for all chains. In these plots information is given concerning
the efficiency of the sampling algorithm. Zero treedepth can indicate
extreme curvature and poorly-chosen step size. Treedepth equal to the
maximum treedepth might be a sign of poor adaptation or of a difficult
posterior from which to sample. The former can be resolved by increasing
the warmup time, the latter might be mitigated by reparametrization. For
more information see
\url{https://mc-stan.org/misc/warnings.html\#maximum-treedepth-exceeded}
or
\href{https://mc-stan.org/docs/2_19/reference-manual/hmc-algorithm-parameters.html}{https://mc-stan.org/docs/reference-manual/hmc-algorithm-parameters.html}.

\includegraphics{ShinyStan_diagnostics_report_files/figure-latex/treedepth_plot-1.pdf}

\newpage

\hypertarget{step-size-information}{%
\subsection{Step Size Information}\label{step-size-information}}

These are plots of the \emph{integrator step size per chain} (x-axis)
against the \emph{log-posterior} (y-axis top panel) and against the
\emph{acceptance statistic} (y-axis bottom panel) of the sampling
algorithm. If the step size is too large, the integrator will be
inaccurate and too many proposals will be rejected. If the step size is
too small, the many small steps lead to long simulation times per
interval. Thus the goal is to balance the acceptance rate between these
extremes. Good plots will show full exploration of the log-posterior and
moderate to high acceptance rates for all chains and step sizes. Bad
plots might show incomplete exploration of the log-posterior and lower
acceptance rates for larger step sizes.

\includegraphics{ShinyStan_diagnostics_report_files/figure-latex/stepsize_plot-1.pdf}

\newpage

\hypertarget{acceptance-information}{%
\subsection{Acceptance Information}\label{acceptance-information}}

These are plots of the \emph{acceptance statistic} (top leftpanel), the
\emph{log-posterior} (top right panel), and, the \emph{acceptance
statistic} (x-axis bottom panel) against the \emph{log-posterior}
(y-axis bottom panel) for all chains. The vertical lines indicate the
mean (solid line) and median (dashed line). A bad plot would show a
relationship between the acceptance statistic and the log-posterior.
This might be indicative of poor exploration of parts of the posterior
which might be might be mitigated by reparametrization or adaptation of
the step size. If many proposals are rejected the integrator step size
might be too large and the posterior might not be fully explored. If the
acceptance rate is very high this might be indicative of inefficient
sampling. The target Metropolis acceptance rate can be set with the
\texttt{adapt\_delta} control option. For more information see
\url{https://mc-stan.org/docs/reference-manual/hmc-algorithm-parameters.html}.

\includegraphics{ShinyStan_diagnostics_report_files/figure-latex/acceptance_plot-1.pdf}

\newpage

\hypertarget{scatter-plots}{%
\subsection{Scatter plots}\label{scatter-plots}}

These are scatter plots of the 10 worst performing parameters in terms
of n\_eff, plotted against \texttt{log-posterior}. The red dots, if
present, indicate divergent transitions. Divergent transitions can
indicate problems for the validity of the results. A good plot would
show no divergent transitions. A bad plot would show divergent
transitions in a systematic patern. For more information see
\url{https://mc-stan.org/misc/warnings.html\#divergent-transitions-after-warmup}.

\includegraphics[width=0.45\linewidth]{ShinyStan_diagnostics_report_files/figure-latex/diverget_scatter_plot-1}
\includegraphics[width=0.45\linewidth]{ShinyStan_diagnostics_report_files/figure-latex/diverget_scatter_plot-2}
\includegraphics[width=0.45\linewidth]{ShinyStan_diagnostics_report_files/figure-latex/diverget_scatter_plot-3}
\includegraphics[width=0.45\linewidth]{ShinyStan_diagnostics_report_files/figure-latex/diverget_scatter_plot-4}
\includegraphics[width=0.45\linewidth]{ShinyStan_diagnostics_report_files/figure-latex/diverget_scatter_plot-5}
\includegraphics[width=0.45\linewidth]{ShinyStan_diagnostics_report_files/figure-latex/diverget_scatter_plot-6}
\includegraphics[width=0.45\linewidth]{ShinyStan_diagnostics_report_files/figure-latex/diverget_scatter_plot-7}
\includegraphics[width=0.45\linewidth]{ShinyStan_diagnostics_report_files/figure-latex/diverget_scatter_plot-8}
\includegraphics[width=0.45\linewidth]{ShinyStan_diagnostics_report_files/figure-latex/diverget_scatter_plot-9}
\includegraphics[width=0.45\linewidth]{ShinyStan_diagnostics_report_files/figure-latex/diverget_scatter_plot-10}

\newpage

\hypertarget{autocorrelation}{%
\subsection{Autocorrelation}\label{autocorrelation}}

These are autocorrelation plots of the 10 worst performing parameters in
terms of n\_eff. The autocorrelation expresses the dependence between
the samples of a Monte Carlo simulation. With higher dependence between
the draws, more samples are needed to obtain the same effective sample
size. High autocorrelation can sometimes be remedied by
reparametrization of the model.

\includegraphics[width=0.45\linewidth]{ShinyStan_diagnostics_report_files/figure-latex/autocorrelation_plot-1}
\includegraphics[width=0.45\linewidth]{ShinyStan_diagnostics_report_files/figure-latex/autocorrelation_plot-2}
\includegraphics[width=0.45\linewidth]{ShinyStan_diagnostics_report_files/figure-latex/autocorrelation_plot-3}
\includegraphics[width=0.45\linewidth]{ShinyStan_diagnostics_report_files/figure-latex/autocorrelation_plot-4}
\includegraphics[width=0.45\linewidth]{ShinyStan_diagnostics_report_files/figure-latex/autocorrelation_plot-5}
\includegraphics[width=0.45\linewidth]{ShinyStan_diagnostics_report_files/figure-latex/autocorrelation_plot-6}
\includegraphics[width=0.45\linewidth]{ShinyStan_diagnostics_report_files/figure-latex/autocorrelation_plot-7}
\includegraphics[width=0.45\linewidth]{ShinyStan_diagnostics_report_files/figure-latex/autocorrelation_plot-8}
\includegraphics[width=0.45\linewidth]{ShinyStan_diagnostics_report_files/figure-latex/autocorrelation_plot-9}
\includegraphics[width=0.45\linewidth]{ShinyStan_diagnostics_report_files/figure-latex/autocorrelation_plot-10}

\newpage

\hypertarget{trance-plots}{%
\subsection{Trance Plots}\label{trance-plots}}

These are trace plots of the 10 worst performing parameters in terms of
n\_eff, for all chains. Trace plots provide a visual way to inspect
sampling behavior and assess mixing across chains. The iteration number
(x-axis) is plotted against the parameter value at that iteration
(y-axis). Divergent transitions are marked on the x-axis. A good plot
shows chains that move swiftly through the parameter space and all
chains that explore the same parameter space without any divergent
transitions. A bad plot shows chains exploring different parts of the
parameter space, this is a sign of non-convergence. If there are
divergent transitions, looking at the parameter value related to these
iterations might provide information about the part of the parameter
space that is difficult to sample from. Slowly moving chains are
indicative of high autocorrelation or small integrator step size, both
of which relate to ineffective sampling and lower effective sample sizes
for the parameter.

\includegraphics[width=0.45\linewidth]{ShinyStan_diagnostics_report_files/figure-latex/trace_plot-1}
\includegraphics[width=0.45\linewidth]{ShinyStan_diagnostics_report_files/figure-latex/trace_plot-2}
\includegraphics[width=0.45\linewidth]{ShinyStan_diagnostics_report_files/figure-latex/trace_plot-3}
\includegraphics[width=0.45\linewidth]{ShinyStan_diagnostics_report_files/figure-latex/trace_plot-4}
\includegraphics[width=0.45\linewidth]{ShinyStan_diagnostics_report_files/figure-latex/trace_plot-5}
\includegraphics[width=0.45\linewidth]{ShinyStan_diagnostics_report_files/figure-latex/trace_plot-6}
\includegraphics[width=0.45\linewidth]{ShinyStan_diagnostics_report_files/figure-latex/trace_plot-7}
\includegraphics[width=0.45\linewidth]{ShinyStan_diagnostics_report_files/figure-latex/trace_plot-8}
\includegraphics[width=0.45\linewidth]{ShinyStan_diagnostics_report_files/figure-latex/trace_plot-9}
\includegraphics[width=0.45\linewidth]{ShinyStan_diagnostics_report_files/figure-latex/trace_plot-10}

\newpage

\hypertarget{rank-plots}{%
\subsection{Rank Plots}\label{rank-plots}}

These are rank plots of the 10 worst performing parameters in terms of
n\_eff, for all chains. Rank histograms visualize how the values from
the chains mix together in terms of ranking. An ideal plot would show
the rankings mixing or overlapping in a uniform distribution. See
\href{https://arxiv.org/abs/1903.08008}{Vehtari et al.~(2019)} for
details.

\includegraphics{ShinyStan_diagnostics_report_files/figure-latex/rank_plot-1.pdf}
\includegraphics{ShinyStan_diagnostics_report_files/figure-latex/rank_plot-2.pdf}
\includegraphics{ShinyStan_diagnostics_report_files/figure-latex/rank_plot-3.pdf}
\includegraphics{ShinyStan_diagnostics_report_files/figure-latex/rank_plot-4.pdf}
\includegraphics{ShinyStan_diagnostics_report_files/figure-latex/rank_plot-5.pdf}
\includegraphics{ShinyStan_diagnostics_report_files/figure-latex/rank_plot-6.pdf}
\includegraphics{ShinyStan_diagnostics_report_files/figure-latex/rank_plot-7.pdf}
\includegraphics{ShinyStan_diagnostics_report_files/figure-latex/rank_plot-8.pdf}
\includegraphics{ShinyStan_diagnostics_report_files/figure-latex/rank_plot-9.pdf}
\includegraphics{ShinyStan_diagnostics_report_files/figure-latex/rank_plot-10.pdf}


\end{document}
